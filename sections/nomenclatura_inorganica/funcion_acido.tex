\subsection*{Función ácido}
\subsubsection*{Ácidos oxácidos}
\begin{Theorem*} {Ácido oxácido}
	\begin{figure}[H]
		\centering
		\begin{tikzpicture}
			\node at (0,3) {$\text{Anhídrido} + \mathrm{H}_2\mathrm{O} \rightarrow \text{Oxácido}$};
			\draw[<-] (-2,2.7) -- (-2,2.5);
			\draw (-2,2.5) -- (-4,2.5);
			\draw (-4,2.5) -- (-4, 1.8);
			\draw[->] (-4,1.8) -- (-3.6, 1.8);
			\draw (-3.5,2.2) -- (-3.5, 1.8);
			\node[anchor=mid west] at (-3.5,2) {- Cualquier anhídrido};
		\end{tikzpicture}
	\end{figure}
	$$\ch{ANH + H_2O ->}\mathrm{H}_x\mathrm{NM}\mathrm{O}_y$$
	o también (forma rápida):
	$$ \mathrm{H}^{+2}\mathrm{NM}^{+n}\mathrm{O}^{-2} \rightarrow \mathrm{H}_1\mathrm{NM}\mathrm{O}_{\frac{n+1}{2}} \leftrightarrow \text{(n es impar)}$$
	$$ \mathrm{H}^{+2}\mathrm{NM}^{+n}\mathrm{O}^{-2} \rightarrow \mathrm{H}_2\mathrm{NM}\mathrm{O}_{\frac{n+2}{2}}\leftrightarrow\text{(n es par)}$$
\end{Theorem*}
\noindent Ejemplos
\begin{align*}
	&\ch{"\ox{+1,H}" "\ox{+1,Cl}" "\ox{-2,O}" -> HClO}\left\{\begin{array}{l}
		\text{T: ácido hipocloroso}
	\end{array}\right. \\
	&\ch{"\ox{+1,H}" "\ox{+3,Cl}" "\ox{-2,O}" -> HClO2}\left\{\begin{array}{l}
		\text{T: ácido cloroso}
	\end{array}\right. \\
	&\ch{"\ox{+1,H}" "\ox{+5,Cl}" "\ox{-2,O}" -> HClO}\left\{\begin{array}{l}
		\text{T: ácido clórico}
	\end{array}\right. \\
	&\ch{"\ox{+1,H}" "\ox{+7,Cl}" "\ox{-2,O}" -> HClO4}\left\{\begin{array}{l}
		\text{T: ácido perclórico}
	\end{array}\right. \\
	&\ch{"\ox{+1,H}" "\ox{+2,Se}" "\ox{-2,O}" -> H2Se2}\left\{\begin{array}{l}
		\text{T: ácido hiposelenioso}
	\end{array}\right. \\
	&\ch{"\ox{+1,H}" "\ox{+4,Se}" "\ox{-2,O}" -> H2Se3}\left\{\begin{array}{l}
		\text{T: ácido selenioso}
	\end{array}\right. \\
	&\ch{"\ox{+1,H}" "\ox{+6,Se}" "\ox{-2,O}" -> H2Se4}\left\{\begin{array}{l}
		\text{T: ácido selénico}
	\end{array}\right. \\
	&\ch{"\ox{+1,H}" "\ox{+7,Mn}" "\ox{-2,O}" -> HMno4}\left\{\begin{array}{l}
		\text{T: ácido permangánico}
	\end{array}\right. \\
	&\ch{"\ox{+1,H}" "\ox{+6,Mn}" "\ox{-2,O}" -> H2Mn4}\left\{\begin{array}{l}
		\text{T: ácido mangánico}
	\end{array}\right. \\
	&\ch{"\ox{+1,H}" "\ox{+6,Cr}" "\ox{-2,O}" -> H2Cr4}\left\{\begin{array}{l}
		\text{T: ácido crómico}
	\end{array}\right. \\
	&\ch{"\ox{+1,H}" "\ox{+5,Bi}" "\ox{-2,O}" -> HBiO3}\left\{\begin{array}{l}
		\text{T: ácido bismútico}
	\end{array}\right. \\
	&\ch{"\ox{+1,H}" "\ox{+3,N}" "\ox{-2,O}" -> HNO2}\left\{\begin{array}{l}
		\text{T: ácido nitroso}
	\end{array}\right. \\
	&\ch{"\ox{+1,H}" "\ox{+5,N}" "\ox{-2,O}" -> HNO3}\left\{\begin{array}{l}
		\text{T: ácido nítrico}
	\end{array}\right. \\
	&\ch{"\ox{+1,H}" "\ox{+3,B}" "\ox{-2,O}" -> HBO2}\left\{\begin{array}{l}
		\text{T: ácido bórico} \\
		\text{T: ácido metabórico}
	\end{array}\right.
	\intertext{casos especiales:}
	&\ch{"\ox{+1,N2}" "\ox{-2,O5}" + H2O -> H2N2O2}\left\{\begin{array}{l}
		\text{T: ácido hiponitroso}
	\end{array}\right. \\
	&\ch{"\ox{+1,P2}" "\ox{-2,O}" + 3 H2O -> H3PO2}\left\{\begin{array}{l}
		\text{T: ácido hipofosforoso}
	\end{array}\right. \\
	&\ch{3 "\ox{+3,B2}" "\ox{-2,O3}" +  H2O -> H2B4O7}\left\{\begin{array}{l}
		\text{T: ácido tetraborico}
	\end{array}\right.
	\intertext{casos especiales del P, As, Sb (ácidos polidratados)}
	&\ch{3 "\ox{+5,P_2}" "\ox{-2,O5}" +  H2O -> HPO3}\left\{\begin{array}{l}
		\cancel{\text{T: ácido fosfórico}}\\
		\text{T: ácido metafosfórico}
	\end{array}\right.
	&\ch{3 "\ox{+5,As_2}" "\ox{-2,O5}" +  H2O -> HAsO3}\left\{\begin{array}{l}
		\cancel{\text{T: ácido arsénico}}\\
		\text{T: ácido metaarsénico}
	\end{array}\right. \\
	&\ch{3 "\ox{+5,Sb_2}" "\ox{-2,O5}" +  H2O -> HSbO3}\left\{\begin{array}{l}
		\cancel{\text{T: ácido antimónico}}\\
		\text{T: ácido metaantimónico}
	\end{array}\right.
\end{align*}
\subsubsection*{Ácidos polihidratados}
\begin{Theorem*} {Ácido polihidratado}
	\begin{figure}[H]
		\centering
		\begin{tikzpicture}
			\node at (0,3) {$\text{Anhídrido} + \mathrm{H}_2\mathrm{O} \rightarrow \text{Oxácido}$};
			\draw[<-] (-2,2.7) -- (-2,2.5);
			\draw (-2,2.5) -- (-4,2.5);
			\draw (-4,2.5) -- (-4, 1.8);
			\draw[->] (-4,1.8) -- (-3.6, 1.8);
			\draw (-3.5,2.2) -- (-3.5, 1.8);
			\node[anchor=mid west] at (-3.5,2) {- 1, 2, 3};
		\end{tikzpicture}
	\end{figure}
	\begin{center}
		\begin{tabularx}{9cm}{|X m{3cm}|X m{3cm}|}
			\hline
			\multicolumn{2}{|c|}{valencia impar} & \multicolumn{2}{|c|}{valencia par} \\ \hline
			meta & \textbf{1} ANH + \textbf{1} \ch{H2O} & meta & \textbf{1} ANH + \textbf{1} \ch{H2O} \\
			piro & \textbf{1} ANH + \textbf{2} \ch{H2O} & orto & \textbf{1} ANH + \textbf{2} \ch{H2O} \\
			orto & \textbf{1} ANH + \textbf{3} \ch{H2O} & piro & \textbf{2} ANH + \textbf{3} \ch{H2O}  \\ \hline
		\end{tabularx}
	\end{center}
\end{Theorem*}
\noindent Ejemplos:
\begin{align*}
	&\ch{"\ox{+5,As2}" "\ox{-2,O5}" + H2O -> HAsO3}\left\{\begin{array}{l}
		\text{T: ácido metaarsénico}
	\end{array}\right. \\
	&\ch{"\ox{+5,As2}" "\ox{-2,O5}" + 2 H2O -> H4As2O7}\left\{\begin{array}{l}
		\text{T: ácido piroarsénico}
	\end{array}\right. \\
	&\ch{"\ox{+5,As2}" "\ox{-2,O5}" + 3 H2O -> H3PO4}\left\{\begin{array}{l}
		\text{T: ácido \cancel{orto}arsénico} \\
		\text{T: ácido arsénico}
	\end{array}\right. \\
	&\ch{"\ox{+5,P2}" "\ox{-2,O5}" + H2O -> HPO3}\left\{\begin{array}{l}
		\text{T: ácido metafosfórico}
	\end{array}\right. \\
	&\ch{"\ox{+5,P2}" "\ox{-2,O5}" + 2 H2O -> H4P2O7}\left\{\begin{array}{l}
		\text{T: ácido pirofosfórico}
	\end{array}\right. \\
	&\ch{"\ox{+5,P2}" "\ox{-2,O5}" + 3 H2O -> H3PO4}\left\{\begin{array}{l}
		\text{T: ácido \cancel{orto}fosfórico} \\
		\text{T: ácido fosfórico}
	\end{array}\right. \\
	&\ch{"\ox{+4,Si}" "\ox{-2,O2}" + H2O -> H2SiO3}\left\{\begin{array}{l}
		\text{T: ácido metasilícico}
	\end{array}\right. \\
	&\ch{"\ox{+4,Si}" "\ox{-2,O2}" + 2 H2O -> H4SiO4}\left\{\begin{array}{l}
		\text{T: ácido \cancel{orto}silícico} \\
		\text{T: ácido silícico}
	\end{array}\right. \\
	&\ch{2 "\ox{+5,As2}" "\ox{-2,O5}" + 3 H2O -> H6Si2O7}\left\{\begin{array}{l}
		\text{T: ácido pirosilícico}
	\end{array}\right. \\
\end{align*}
\subsubsection*{Diácidos}
\begin{Theorem*} {Diácidos}
	\begin{figure}[H]
		\centering
		\begin{tikzpicture}
			\node at (0,3) {$\text{2 Anhídrido} + \ch{H2O} \rightarrow \text{diácido}$};
			\draw[<-] (-2,2.7) -- (-2,2.5);
			\draw (-2,2.5) -- (-4,2.5);
			\draw (-4,2.5) -- (-4, 1.8);
			\draw[->] (-4,1.8) -- (-3.6, 1.8);
			\draw (-3.5,2.2) -- (-3.5, 1.3);
			\node[anchor=mid west] at (-3.5,2) {- Anfígenos con +6};
			\node[anchor=mid west] at (-3.5,1.5) {- Anfóteros con +6};
		\end{tikzpicture}
	\end{figure}
	$$\ch{2 ANH + H_2O ->}\mathrm{H}_2\mathrm{ANH}_2\mathrm{O}_7$$
\end{Theorem*}
\noindent Ejemplos:
\begin{align*}
	&\ch{2 "\ox{+6,S}" "\ox{-2,O3}" + H2O -> H2S2O7}\left\{\begin{array}{l}
		\text{T: ácido difulfúrico}
	\end{array}\right. \\
	&\ch{2 "\ox{+6,Mn}" "\ox{-2,O3}" + H2O -> H2Mn2O7}\left\{\begin{array}{l}
		\text{T: ácido dimangánico}
	\end{array}\right. \\
\end{align*}
\subsubsection*{Peroxiácidos}
\begin{Theorem*} {Peroxiácidos}
	\begin{figure}[H]
		\centering
		\begin{tikzpicture}
			\node at (0,3) {$\text{Anhídrido} + \ch{H2O2} \rightarrow \text{Peroxiácidos}$};
			\draw[<-] (-2,2.7) -- (-2,2.5);
			\draw (-2,2.5) -- (-4,2.5);
			\draw (-4,2.5) -- (-4, 1.8);
			\draw[->] (-4,1.8) -- (-3.6, 1.8);
			\draw (-3.5,2.2) -- (-3.5, 1.8);
			\node[anchor=mid west] at (-3.5,2) {- Cualquier anhídrido};
		\end{tikzpicture}
	\end{figure}
	$$\ch{ANH + H_2O_2 ->}\mathrm{H}_m\mathrm{ANH}_n\mathrm{O}_p$$
\end{Theorem*}
\noindent Ejemplos:
\begin{align*}
	&\ch{"\ox{+6,S}" "\ox{-2,O3}" + H2O2 -> H2SO5}\left\{\begin{array}{l}
		\text{T: ácido peroxisulfúlrico}
	\end{array}\right. \\
	&\ch{"\ox{+6,Cr}" "\ox{-2,O3}" + H2O2 -> H2CrO5}\left\{\begin{array}{l}
		\text{T: ácido peroxicrómico}
	\end{array}\right. \\
	&\ch{"\ox{+7,I2}" "\ox{-2,O7}" + H2O2 -> H2I2O9}\left\{\begin{array}{l}
		\text{T: ácido peroxiperyódico}
	\end{array}\right. \\
	&\ch{2 "\ox{+6,S}" "\ox{-2,O3}" + H2O2 -> H2S2O8}\left\{\begin{array}{l}
		\text{T: ácido peroxidisulfúrico}
	\end{array}\right. \\
\end{align*}
\subsubsection*{Tioácidos}
\begin{Theorem*} {Tioácidos}
	\begin{figure}[H]
		\centering
		\begin{tikzpicture}
			\node at (0,3) {$\text{oxácido} - \mathrm{nO} + \mathrm{nS} \rightarrow \text{Tioácido}$};
			\draw[<-] (-2,2.7) -- (-2,2.5);
			\draw (-2,2.5) -- (-4,2.5);
			\draw (-4,2.5) -- (-4, 1.8);
			\draw[->] (-4,1.8) -- (-3.6, 1.8);
			\draw (-3.5,2.2) -- (-3.5, 1.8);
			\node[anchor=mid west] at (-3.5,2) {- Cualquier oxácido};
		\end{tikzpicture}
	\end{figure}
	$$\ch{ANH + H_2O_2 ->}\mathrm{H}_m\mathrm{ANH}_n\mathrm{O}_p$$
\end{Theorem*}
\noindent Ejemplos:
\begin{align*}
	&\ch{HNO3 - O + S -> HNO2S}\left\{\begin{array}{l}
		\text{T: ácido tionítrico}
	\end{array}\right. \\
	&\ch{HNO3 - 2 O + 2 S -> HNOS2}\left\{\begin{array}{l}
		\text{T: ácido ditionítrico}
	\end{array}\right. \\
	&\ch{HNO3 - 3 O + 3 S -> HNS3}\left\{\begin{array}{l}
		\text{T: ácido sulfonítrico}
	\end{array}\right. \\
\end{align*}
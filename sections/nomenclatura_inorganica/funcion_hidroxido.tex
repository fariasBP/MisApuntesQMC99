\subsection*{Función hidróxido}
Son compuestos ternarios que se caracterizan por que poseen el ion hidróxido o hidroxilo $\ch{(OH)^{-1}}$.
\subsubsection*{Hidróxidos}
Son compuestas que surgen de la combinación de un metal y el ion hidróxido.
\begin{Theorem*} {Hidróxidos}
	\begin{figure}[H]
		\centering
		\begin{tikzpicture}
			\node at (0,3) {$\text{Metal} + \mathrm{(OH)}^{-1} \rightarrow \text{Hidróxido}$};
			\draw[<-] (-2,2.7) -- (-2,2.5);
			\draw (-2,2.5) -- (-4,2.5);
			\draw (-4,2.5) -- (-4, 1.8);
			\draw[->] (-4,1.8) -- (-3.6, 1.8);
			\draw (-3.5,2.2) -- (-3.5, 1.3);
			\node[anchor=mid west] at (-3.5,2) {- Cualquier no metal};
			\node[anchor=mid west] at (-3.5,1.5) {- Anfóteros con +2 +3};
		\end{tikzpicture}
	\end{figure}
	$$\ch{M^{+} + (OH)^{-1} ->}\mathrm{M}\mathrm{(OH)}_x$$
\end{Theorem*}
\noindent Ejemplos:
\begin{align*}
	&\ch{"\ox{+1,Li}" + "\ox{-1,OH}" -> LiOH} \ \text{\{ T: hidróxido lítico} \\
	&\ch{"\ox{+1,Hg}" + "\ox{-1,OH}" -> HgOH} \ \text{\{ T: hidróxido mercurioso} \\
	&\ch{"\ox{+2,Hg}" + "\ox{-1,OH}" -> Hg(OH)2} \ \text{\{ T: hidróxido mercúrico} \\
	&\ch{"\ox{+3,Mn}" + "\ox{-1,OH}" -> Mn(OH)3} \ \text{\{ T: hidróxido mangánico} \\
	&\ch{"\ox{+2,Mn}" + "\ox{-1,OH}" -> Mn(OH)2} \ \text{\{ T: hidróxido manganoso} \\
	&\ch{"\ox{+4,Sn}" + "\ox{-1,OH}" -> Sn(OH)4} \ \text{\{ T: hidróxido estánico} \\
	&\ch{"\ox{+2,Sn}" + "\ox{-1,OH}" -> Hg(OH)2} \ \text{\{ T: hidróxido estanoso}
	\intertext{casos especiales:}
	&\ch{"\ox{+2,Hg2}" + "\ox{-1,OH}" -> Hg2(OH)2} \ \text{\{ T: hidróxido mercúrioso} (?)
\end{align*}

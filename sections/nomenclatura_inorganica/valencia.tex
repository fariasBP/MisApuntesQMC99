\subsection*{Valencia}
Es la capacidad de combinación que posee el átomo de un elemento para formar compuestos. Se representa por un numero sin signo llamado numero de valencia.
\end{multicols}
\subsubsection*{Tabla de valencias}
\textit{\textbf{Metales}}
\begin{center}
\begin{tabularx}{8cm}{|X|X|X|}
	\hline
	\multicolumn{3}{|c|}{Monovalentes} \\ \hline
	\multicolumn{2}{|c|}{Nombre} & ico \\ \hline
	Litio & Li & +1 \\
	Sodio & Na & +1 \\
	Potasio & K & +1 \\
	Rubidio & Rb & +1 \\
	Cesio & Cs & +1  \\
	Francio & Fr & +1  \\
	Plata & Ag & +1  \\
	Amonio & $\ch{NH4}$ & +1  \\ \hline
\end{tabularx}
\hspace{0.5cm}
\begin{tabularx}{8cm}{|X|X|X|}
	\hline
	\multicolumn{3}{ |c| }{Bivalentes} \\ \hline
	\multicolumn{2}{|c|}{Nombre} & ico \\ \hline
	Berilio & Be & +2 \\
	Magnesio & Mg & +2 \\
	Calcio & Ca & +2 \\
	Estroncio & Sr & +2 \\
	Bario & Ba & +2 \\
	Radio & Ra & +2 \\
	Cadmio & Cd & +2 \\
	Zinc & Zn & +2 \\ \hline
\end{tabularx}
\end{center}
\begin{center}
\begin{tabularx}{8cm}{|X|X|X|}
	\hline
	\multicolumn{3}{ |c| }{Trivalentes} \\ \hline
	\multicolumn{2}{|c|}{Nombre} & ico \\ \hline
	Aluminio & Al & +3 \\
	Galio & Ga & +3 \\
	Indio & In & +3 \\ \hline
\end{tabularx}
\hspace{0.5cm}
\begin{tabularx}{8cm}{|X|X|X X|}
	\hline
	\multicolumn{4}{ |c| }{Monobivalentes} \\ \hline
	\multicolumn{2}{|c|}{Nombre} & oso & ico \\ \hline
	Cobre & Cu & +1 & +2 \\
	Mercurio & Hg & +1 & +2 \\ \hline
\end{tabularx}
\end{center}
\begin{center}
\begin{tabularx}{8cm}{|X|X|X X|}
	\hline
	\multicolumn{4}{ |c| }{Monotrivalentes} \\ \hline
	\multicolumn{2}{|c|}{Nombre} & oso & ico \\ \hline
	Oro & Au & +1 & +3 \\
	Talio & Tl & +1 & +3 \\ \hline
\end{tabularx}
\hspace{0.5cm}
\begin{tabularx}{8cm}{|X|X|X X|}
	\hline
	\multicolumn{4}{ |c| }{Bitrivalentes} \\ \hline
	\multicolumn{2}{|c|}{Nombre} & oso & ico \\ \hline
	Hierro & Fe & +2 & +3 \\
	Cobalto & Co & +2 & +3 \\
	Niquel & Ni & +2 & +3 \\ \hline
\end{tabularx}
\end{center}
\begin{center}
\begin{tabularx}{8cm}{|X|X|X X|}
	\hline
	\multicolumn{4}{|c|}{Bitetravalentes} \\ \hline
	\multicolumn{2}{|c|}{Nombre} & oso & ico \\ \hline
	Estaño & Sn & +2 & +4 \\
	Plomo & Pb & +2 & +4 \\
	Platino & Pt & +2 & +4 \\
	Paladio & Pd & +2 & +4 \\ \hline
\end{tabularx}
\end{center}
\textit{\textbf{No metales}}
\begin{center}
\begin{tabularx}{12cm}{| X | X |X X l l X| }
	\hline
	\multicolumn{7}{ |c| }{Halogenoides} \\ \hline
	\multicolumn{2}{|c|}{Nombre} & hídrico & hipo-oso & oso & ico & per-ico \\ \hline
	Fluor & F & -1 & & & & \\
	Cloro & Cl & -1 & +1 & +3 & +5 & +7 \\
	Bromo & Br & -1 & +1 & +3 & +5 & +7 \\
	Yodo & I & -1 & +1 & +3 & +5 & +7 \\ \hline
\end{tabularx}
\end{center}
\begin{center}
\begin{tabularx}{12cm}{| X | X |X X l l| }
	\hline
	\multicolumn{6}{ |c| }{Anfigenoides} \\ \hline
	\multicolumn{2}{|c|}{Nombre} & hídrico & hipo-oso & oso & ico \\ \hline
	Oxigeo & O & -2 & & & \\
	Azufre & S & -2 & +2 & +4 & +6 \\
	Selenio & Se & -2 & +2 & +4 & +6 \\
	Teluro & Te & -2 & +2 & +4 & +6 \\ \hline
\end{tabularx}
\end{center}
\begin{center}
\begin{tabularx}{14cm}{| X | X |X X l l l l|}
	\hline
	\multicolumn{8}{ |c| }{Nitrogenoides} \\ \hline
	\multicolumn{2}{|c|}{Nombre} & hídrico & hipo-oso & esp & oso & esp & ico \\ \hline
	Nitrogeno & N & -3 & +1 & +2 & +3 & +4 & +5 \\
	Fosforo & P & -3 & & & +3 & & +5 \\
	Antimonio & Sb & -3 & & & +3 & & +5 \\
	Arsenico & As & -3 & & & +3 & & +5 \\
	Boro & B & -3 & & & & & +3 \\ \hline
\end{tabularx}
\end{center}
\begin{center}
\begin{tabularx}{10cm}{| X | X |X X l| }
	\hline
	\multicolumn{5}{ |c| }{Carbonoides} \\ \hline
	\multicolumn{2}{|c|}{Nombre} & hídrico & oso & ico \\ \hline
	Carbono & C & -4 & +2 & +4 \\
	Silicio & Si & -4 & & +4 \\ \hline
\end{tabularx}
\end{center}
\textit{\textbf{Polivalentes o anfóteros}} 
\begin{center}
\begin{tabularx}{13.2cm}{| l | l |ll|X X X X| }
	\hline
	\multicolumn{8}{|c|}{Anfóteros} \\ \hline
	\multicolumn{2}{|c|}{\multirow{2}{*}{Nombre}} & \multicolumn{2}{c|}{metal} & \multicolumn{4}{c|}{no metal} \\
	\multicolumn{2}{|c|}{} & oso & ico & hipo-oso & oso & ico & per-ico \\ \hline
	Manganeso & Mn & +2 & +3 & & +4 & +6 & +7 \\
	Molibdeno & Mo & +2 & +3 & +4 & +5 & +6 & \\
	Wolfranio & W & +2 & +3 & +4 & +5 & +6 & \\
	Vanadio & V & +2 & +3 & & +4 & +5 & \\
	Titanio & Ti & +2 & +3 & & & +4 & \\
	Cromo & Cr & +2 & +3 & & & +6 & \\
	Uranio & U & & +3 & +4 & +5 & +6 & \\
	Bismuto & Bi & & +3 & & & +5 & \\ \hline
\end{tabularx}
\end{center}
o también de la siguiente manera:
\begin{center}
	\begin{tabularx}{10cm}{|l|X|cc|cccc|}
		\hline
		\multicolumn{2}{|c|}{Nombre} & \multicolumn{6}{c|}{valencias} \\ \hline
		Manganeso & Mn & +2 & +3 & +4 & & +6 & +7 \\
		Molibdeno & Mo & +2 & +3 & +4 & +5 & +6 & \\
		Wolfranio & W & +2 & +3 & +4 & +5 & +6 & \\
		Vanadio & V & +2 & +3 & & +5 & & \\
		Titanio & Ti & +2 & +3 & +4 & & & \\
		Cromo & Cr & +2 & +3 & & & +6 & \\
		Uranio & U & & +3 & +4 & +5 & +6 & \\
		Bismuto & Bi & & +3 & & +5 & & \\ \hline
	\end{tabularx}
\end{center}
\begin{multicols}{2}
Los anfóteros suelen ser elementos muy difíciles de recordar debido a su cantidad de valencias, para facilitar esto podemos indicar algunos puntos importantes:
\begin{itemize}
	\item Las valencias \textbf{+2}, \textbf{+3} siempre son para metales y son: \textbf{oso} e \textbf{ico} respectivamente.
	\item La valencia \textbf{+7} siempre sera tendrá el sufijo-prefijo \textbf{per-ico}. 
	\item Si el elemento tiene 4 valencias pertenecientes al no metal (o al metal) estas en forma ascendete deben ser \textbf{hipo-oso}, \textbf{oso}, \textbf{ico} y \textbf{per-ico}; si tiene 3 deben ser \textbf{hipo-oso}, \textbf{oso} e \textbf{ico}; si tiene 2 valencia deber \textbf{oso} e \textbf{ico}; y si tiene una esta debe ser \textbf{ico}. Note que mientras menos valencia, está tiende a \textbf{ico} ( cabe resaltar que las valencias del metal siempre son 2).
\end{itemize}
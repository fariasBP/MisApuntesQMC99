\section{Composición cuantitativa de la sustancia}
\subsection*{Átomo}
El atomo es un sistema energetico electricamente neutro que representa una parte interna llamada nucleo y otra zona extranuclear o nube electronica.
\subsubsection*{Definiciones básicas}
\textbf{\textit{Estado natural de un átomo, }} Es la condición en la que todo átomo es neutro, entonces:
$$ \#p^+=\#e^- $$
\textbf{\textit{Numero atómico (Z), }} Es el numero que indica la cantidad de protones que tiene el átomo en su núcleo:
$$ Z=\#p^+ $$
El numero $Z$ representa la identidad del elemento, pues \textit{todos los átomos de un mismo elemento tienen el mismo numero de protones en el núcleo}. Es llamado también carga nuclear.
\textbf{\textit{Numero de masa (A), }} Es el número que indica la suma de protones ($Z$)y neutrones ($N$) de un átomo. También se le llama numero de nucleones fundamentales del átomo y se le considera como la masa atómica del mismo.
$$ A=Z+N $$
Por lo tanto:
\begin{Theorem*} {Notación de un átomo}
	Un átomo se denota de la siguiente manera:
	$$ ^A_Z E^\pm $$
	donde:
	\begin{itemize}
		\item $E$: átomo
		\item $Z$: numero atómico o carga nuclear
		\item $N$: numero de masa o masa atómica
		\item $q$: carga del atomo
	\end{itemize}
\end{Theorem*}
\textbf{\textit{Carga de atomo (q),}}
$$ \#e^-=Z-(q) $$
\subsection*{Masa Isotópica}
\subsubsection*{El átomo patrón}
El carbono es un elemento crucial e importante, esto es debido a sus propiedades y aplicaciones. El isotopo del carbon mas comun es el carbono 12, y como veremos a continuacion las unidades se basan en propiedades de este isotopo.
\subsubsection*{Masa isotópica y el espectrómetro de masas}
Se sabe que un elemento químico es una mezcla de isótopos, ademas la masa atómica relativa de un isótopo se llama \textbf{masa isotópica}, el cual se mide con un aparato llamado espectrómetro de masas. Para ello se fija convencionalmente una unidad, llamada unidad de masa atomico o abreviado como \textbf{uma}
\subsubsection*{Unidad de masa atómica (\textbf{uma})}
Uma es unidad de medida que represente la masa de un atomo, conveniente mente un uma viene siendo la doceva parte de la masa del isotopo Carbono-12
\begin{Theorem*} {Uma}
	La unidad de masa atomica (uma), es la doceava parte $\left(\frac{1}{12}\right)$ de la masa del isotopo carbono-12 $\left(\isotope*{12,C}\right)$
	$$ 1 [\text{uma}] = 1.66\times10^{-24}[\text{g}] $$
	$$ 1[\text{g}]=6.022\times10^23 [\text{uma}] $$
\end{Theorem*} 
\textbf{\textit{¿Que diferencia hay entre numero de masa ($A$)y masa isotopica, }} El numero de masa es siempre un numero entero ($A$= numero protones + numero de neutrones), mientras que la masa isotopica es un numero decimal y se expresa en uma. Los calculos matematicos, ¿con que masa istopica se relizan? ¿con la del isotpo mas abundante? ¡No!, se realiza con la masa atomica promedio.
\subsection*{Masa atómica y masa molecular}
\subsubsection*{Masa atómica promedio MA}
También denominado \textbf{masa atómica (MA)} o mas comúnmente \textbf{peso atómico (PA)}. Está representa la masa atomica relativo promedio del atomo de un elemento, es decir un promedio ponderado de las masas atomicas relativas de los isotpos de un elemento.
$$ \text{MA}_E=\sum_{i=1}^{n}\frac{A_i\cdot a_i}{a_i} $$
$$ \text{MA}_E=\frac{A_1\cdot a_1+A_2\cdot a_2+\cdots+A_n\cdot a_n}{a_1+a_2+\cdots+a_n} $$
Donde $A_i$ representa la masa isotopica del atomo $i$, y $a_i$ representa la abundancia del atomo $i$. Sabiendo que la suma de la abundacia da como resultado a 100\%, podemos simplificarlo de la siguiente manera:
$$ \text{MA}_E=\frac{A_1\cdot a_1+A_2\cdot a_2+\cdots+A_n\cdot a_n}{100} $$
\textbf{\textit{¿Masa atómica (MA) y Masa atómica promedio, son lo mismo?, }} No en realidad, sin embargo la diferencia es tan pequeña que puede ser despreciable, esto resulta muy útil debido a que tenemos la masa atómica de cualquier elemento en la tabla periódica. \\
La masa atómica de algunos elementos comunes en problemas son:
$$
\begin{array} {|c|c|c|c|c|c|c|c|}
	\hline \text{Elemento} & \ch{H} & \ch{C} & \ch{N} & \ch{O} & \ch{Na} & \ch{P} & \ch{S} \\
	\hline \text{MA} & 1 & 12 & 14 & 16 & 23 & 31 & 32 \\
	\hline
\end{array}
$$
$$
\begin{array} {|c|c|c|c|}
	\hline \text{Elemento} & \ch{Cl} & \ch{K} & \ch{Ca} \\
	\hline \text{MA} & 35.5 & 39 & 40 \\
	\hline
\end{array}
$$
\subsubsection*{Masa molecular}
representa la masa relativa promedio de una molécula de una sustancia covalente. Esta se determina sumando los pesos atómicos de los elementos teniendo en cuenta el numero de átomos de cada  uno en la molécula
$$ \text{MM}_{molecula}= \sum_{i=1}^{n} \alpha_i\text{MA}_i $$
Donde $i$ representa un elemento de la molécula y $\alpha$ la cantidad de átomos de este en la unidad formula.
\begin{Example*} {Ejemplo - Masa molecular}
	\textbf{\textit{Determinar la masa molecular a partir de su formula}} \\
	Determinar la masa molecular de \ch{P4}, \ch{H20}, \ch{H2S04} y \ch{C2H2OH}.
	\begin{enumerate}
		\item $\ch{P4}\Rightarrow\text{MM}_{\ch{P4}}=4\cdot\text{MA}_{\ch{P}}=4\cdot31=124[\text{uma}]$
		\item $\ch{H20}\Rightarrow\text{MM}_{\ch{H20}}=2\cdot\text{MA}_{\ch{H}}+1\cdot\text{MA}_{\ch{O}}=2\cdot1+1\cdot16=18[\text{uma}]$
		\item $\ch{H2S04}\Rightarrow\text{MM}_{\ch{P4}}=2\cdot1+32+4\cdot16=98[\text{uma}]$
		\item $\ch{C2H5OH}\Rightarrow\text{MM}_{\ch{P4}}=2\cdot16+5+16+1=46[\text{uma}]$
	\end{enumerate}
\end{Example*}
\subsection*{El mol}
El mol no es mas que una forma de contar al igual que la docena o millares, para cada una de estas formas de contar cumplen una funcion, por ejemplo la docena lo usamos para contar huevos util porque el comercio de este producto se basa en la cantidad de doce, pero esta forma de contar ya no sera util si contamos granos de cafe, entonces usaremos algo mas grande como los millares, pero si ahora queremos contar atomos necesitaremos una base mas grande, y para esto usaremos el mol.
\begin{Theorem*} {Definicion - Mol}
	Es una unidad de materia de un sistema, que contiene la misma cantidad de unidades elemenatels como atomos hay contenidos en 12 [g] de carbono-12. 
	La cantidad de atomos que hay en nuna muestra de 12 [g] de cargibi es $6.0221367\times10^{23}$ (llamado el número de Avogadro $\mathrm{N_A}$), que con fines practicos se trabaja con $6.022\times10^{23}$
	$$ 1\mathrm{[mol]} = 6.022\times10^{23} \mathrm{[unidades]} $$
\end{Theorem*}
Entonces:
\begin{itemize}
	\item $1\mathrm{[mol]} \ \text{de átomos}=6.022\times1-^{23}$$ \ \text{átomos}$
	\item $1\mathrm{[mol]} \ \text{de moléculas}=6.022\times1-^{23}$$ \ \text{moléculas}$
	\item $1\mathrm{[mol]} \ \text{de electrones}=6.022\times1-^{23}$$ \ \text{electrones}$
	\item $1\mathrm{[mol]} \ \text{de iónes}=6.022\times1-^{23}$$ \ \text{iónes}$
	\item $1\mathrm{[mol]} \ \text{de fotones}=6.022\times1-^{23}$$ \ \text{fotones}$
\end{itemize}
\subsubsection*{Masa molar}
\begin{Theorem*} {Definición - Masa molar}
	La masa molar se define como la masa del mol de una sustancia, denotada por $\bar{\mathrm{M}}$ y expresada en gramos por mol [g/mol], que es numéricamente igual al peso molecular de la misma.
	$$ \bar{\mathrm{M}} \cong \mathrm{MM} $$
\end{Theorem*}
Por ejemplo:
$$ \ch{N2} \rightarrow \bar{\mathrm{M}}=28\mathrm{[g/mol]} \quad \ch{O2} \rightarrow \bar{\mathrm{M}}=32\mathrm{[g/mol]} $$
\subsubsection*{cantidad de sustancia (n)}
\begin{Theorem*} {Definicion - cantidad de sustancia}
	Indica la cantidad de moles contenida de una sustancia en una muestra y se determina segun:
	$$ n=\frac{m}{\bar{\mathrm{M}}} $$
\end{Theorem*}
\begin{Example*} {Ejemplo}
	\textbf{\textit{cantidad de moles}}\\
	Si se tiene 686[g] de $\ch{H2SO4}$, determinar la cantidad de moles, cantidad de atomos y su cantidad de moleculas.
	\begin{flalign*}
		&sol:
		\intertext{- para la masa molar}
		&\mathrm{MM}_{\ch{H2SO4}}=2+32+4(16)=98\mathrm{[uma]} \\
		&\Rightarrow \bar{\mathrm{M}}_{\ch{H2SO4}}=98\mathrm{[g/mol]}
		\intertext{- para cantidad de moles}
		&686\mathrm{[g]}_{\ch{H2SO4}}\times\frac{1\mathrm{[mol]}_{\ch{H2SO4}}}{98\mathrm{[g]}_{\ch{H2SO4}}}=7\mathrm{[mol]}_{\ch{H2SO4}}
		\intertext{- para numero de átomos}
		&686\mathrm{[g]}_{\ch{H2SO4}}\times\frac{1\mathrm{[mol]}_{\ch{H2SO4}}}{98\mathrm{[g]}_{\ch{H2SO4}}} \\
		&\times\frac{6.022\times10^{23} \ \text{átomos}_{\ch{H2SO4}}}{1\mathrm{[mol]}_{\ch{H2SO4}}} \\
		&= 4.2154\times10^{24} \text{ átomos de }\ch{H2SO4}
		\intertext{- para cantidad de moleculas}
		&686\mathrm{[g]}_{\ch{H2SO4}}\times\frac{1\mathrm{[mol]}_{\ch{H2SO4}}}{98\mathrm{[g]}_{\ch{H2SO4}}} \\
		&\times\frac{6.022\times10^{23} \ \text{moléculas}_{\ch{H2SO4}}}{1\mathrm{[mol]}_{\ch{H2SO4}}} \\
		&= 4.2154\times10^{24} \text{ moléculas de }\ch{H2SO4}
	\end{flalign*}
\end{Example*}
\subsubsection*{Volumen molar}
Se define como el volumen, expresado en litro, que presenta 1 modl de un sustancia gaseosa a condiciones normales de presion (1[atm]) y temperatura (273K) e igual a 22.4[l].
$$ 1\mathrm{[mol]}_x \xrightarrow{\text{Ocupa en CN}} 22.4\mathrm{[l]} $$
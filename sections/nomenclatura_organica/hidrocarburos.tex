\subsection*{Hidrocarburos}
\subsubsection*{Hidrocarburos saturados: alcanos o parafinas}
\begin{Theorem*} {Alcanos}
	\begin{gather*}
		\text{Formulación molecular} \\
		\chemfig{C_nH_{2n+2}} \\
		\text{Formulación desarrollada} \\
		\chemfig{C-C} \\
		\text{Formulación desarrollada} \\
		\text{sufijo: "ano"}
	\end{gather*}
\end{Theorem*}
\noindent Ejemplos:\\
%\chemfig{H-C(-[2]H)(-[6]H)-C(=[1]O)-[7]H}
\begin{gather*}
	\chemfig{H-C(-[2]H)(-[6]H)(-[8]H)} \\
	\iupac{Metano} \\
	\chemfig{C-[:30]C}\\
	\iupac{Etano} \\
	\chemfig{C-[:30]C-[:-30]C-[:30]C} \\
	\iupac{Butano} \\
	\chemfig{C-[:30]C-[:-30]C-[:30]C-[:-30]c-[:30]C-[:-30]C} \\
	\iupac{heptano}
\end{gather*}
\subsubsection*{Hidrocarburos insaturados: alquenos, olefinas o etilénicos}
\begin{Theorem*} {Alquenos}
	\begin{gather*}
		\text{Formulación molecular} \\
		\chemfig{C_nH_{2n}} \\
		\text{Formulación desarrollada} \\
		\chemfig{C=C} \\
		\text{Formulación desarrollada} \\
		\text{sufijo: "eno"}
	\end{gather*}
\end{Theorem*}
\noindent Ejemplo: \\
\begin{gather*}
	\chemfig{C(-[:150]H)(-[:270]H)=[:30]C(-[2]H)(-[:-30]H)} \\
	\iupac{Eteno} \\
	\chemfig{C=[:-30]C-[:30]C-[:-30]C-[:30]C}\\
	\iupac{1-Pentano} \\
	\chemfig{C-[:30]C=[:-30]C-[:30]C-[:-30]C} \\
	\iupac{2-Pentano}
\end{gather*}
\subsubsection*{Hidrocarburos insaturados: alquinos o acetilenos}
\begin{Theorem*} {Alquinos}
	\begin{gather*}
		\text{Formulación molecular} \\
		\chemfig{C_nH_{2n-2}} \\
		\text{Formulación desarrollada} \\
		\chemfig{C~C} \\
		\text{Formulación desarrollada} \\
		\text{sufijo: "ino"}
	\end{gather*}
\end{Theorem*}
\noindent Ejemplo: \\
\begin{gather*}
	\chemfig{C(-[-2]H)~[:30]C(-[2]H)} \\
	\iupac{etino} \\
	\chemfig{C~[:30]C-[:-30]C-[:30]C-[:-30]C-[:30]C}\\
	\iupac{1-Hexino} \\
	\chemfig{C-[:30]C~[:-30]C-[:30]C-[:-30]C-[:30]C} \\
	\iupac{2-Hexino} \\
	\chemfig{C-[:30]C-[:-30]C~[:30]C-[:-30]C-[:30]C} \\
	\iupac{3-Hexino}
\end{gather*}
\subsubsection*{Hidrocarburos alicíclicos o cicloalifáticos: cicloalcanos o cicloparafinas}
\begin{Theorem*} {Cicloalcanos}
	\begin{gather*}
		\text{Formulación molecular} \\
		\chemfig{C_nH_{2n}} \\
		\text{Formulación desarrollada} \\
		\chemfig{*3(---)} \\
		\text{Formulación desarrollada} \\
		\text{prefijo: "ciclo" \quad sufijo: "ano"}
	\end{gather*}
\end{Theorem*}
\noindent Ejemplo: \\
\begin{gather*}
	\chemfig{C(-[:150]H)(-[:210]H)*3(-C(-[-2]H)(-[:-30]H)-C(-[2]H)(-[:30]H)-)} \\
	\iupac{Ciclopropano} \\
	\chemfig{C*6(-C-C-C-C-C-)}\\
	\iupac{Ciclohexano} \\
\end{gather*}
\subsubsection*{Sustituyentes o Radicales}
\begin{Theorem*} {Sustituyentes}
	\begin{gather*}
		\text{Formulación desarrollada} \\
		\chemfig{R-R'} \\
		\text{Formulación desarrollada} \\
		\text{sufijo: "il" o "ilo"}
	\end{gather*}
\end{Theorem*}
\noindent \textbf{\textit{Sustituyentes comunes}}
\begin{itemize}
	\item Sustituyentes alcanos
	\begin{itemize}
		\item Metil \quad \chemfig{R-C}
		\item Etil \quad \chemfig{R-[:-30]C-[:30]C}
		\item Propil \quad \chemfig{R-[:-30]C-[:30]C-[:-30]C}
		\item Butil \quad \chemfig{R-[:-30]C-[:30]C-[:-30]C-[:30]C} \\
		\vdots
	\end{itemize}
	\item Sustituyentes ciclicos
	\begin{itemize}
		\item Ciclopropil \quad \chemfig{R-C*3(-C-C-)}
		\item Cicloetil \quad \chemfig{R-C*4(-C-C-C-)}\\
		\vdots
	\end{itemize}
	\item Sustituyentes especiales
	\begin{itemize}
		\item Isopropil \quad \chemfig{R-C(-[:-60]C)(-[:60]C)} \\
		\item Terbutil \quad \chemfig{R-C(-[:30]C)(-C)(-[:-30]C)} \\
		\item Isobutil \quad \chemfig{R-[:-30]C-[:30]C(-[:-30]C)(-[2]C)} \\
		\item Secbutil \quad \chemfig{R-C(-[:-60]C)(-[:60]C-C)}
	\end{itemize}
\end{itemize}
\noindent \textbf{\textit{Regla de nomenclatura}}
\begin{enumerate}
	\item La cadena principal es aquella que contiene el mayor número de enlaces múltiples (mayor número de enlaces dobles y triples).
	\item La cadena principal se numera por el extremo mas cercano a la primera transacciona (al enlace doble o al enlace triple). si el enlace doble y triple son equidistantes, el doble enlace debe tener la menor numeración, es decir, tendrá prioridad sobre el triple enlace.
	\item Los grupos alquilo se nombran en orden alfabético indicando su posición respectiva en la cadena principal, luego se indica la posición del enlace doble (o dobles) con la terminación en, dien, trien, etc. y, finalmente se indica la posición del triiple enlace (o triples) con la terminación ino, diino, triino, etc.
\end{enumerate}
\end{multicols}
\noindent Ejemplos: \\
\begin{gather*}
	\chemfig{CH_3-CH_2-CH(-[-2]CH_2-[-2]CH_2-[-2]CH_3)-CH(-[-2]CH_3)-CH(-[2]CH_2-CH_3)-CH_3} \\
	\iupac{5-etil-3,4-dimetil-octano} \\
	\chemfig{CH_3-C(-[-2]CH_3)=CH-CH(-[2]CH_2-[2]CH_3)-CH_3} \\
	\iupac{2,4-dimetil-2-hexeno} \\
	\chemfig{CH_2=CH-C(=[-2]C(-Br)(-[-2]C(-*4(----))(-[4]H_3C)-[-2]CH2-[-2]C(-CH_2-CH_2-CH_2-CH_3)=[-2]C(-CH_3)-[-2]CH_3))-CH_2-CH_3} \\
	\iupac{4-Bromo-7-butil-4-ciclobutil-3-etil-5,8-dimetil-1,3,7-nonatrieno} \\
	\chemfig{CH_3-C~C-CH(-[-2]CH(-[:-30]CH_3)(-[:-150]CH_3))-CH_2-CH(-[2]C(-[4]CH_3)(-[2]CH_3)(-CH_2-CH_3))-CH_2-CH(-[-2]CH_2-[-2]CH_2-CH_2-CH_3)-C~CH} \\
	\iupac{3-Butil-7isopropil-5-(1,1-dimetilpropil)-1,8-decadiino} \\
\end{gather*}
\begin{multicols} {2}
\noindent \textit{\textbf{Regla de nomenclatura de los cíclicos}}
\begin{enumerate}
	\item Si tiene un sustituyente, se nombra este y luego el nombre del cicloalcano. No se indica la posición del sustituyente ya que se sobreentiende que está en el carbono número uno.
	\item Si tiene dos o más sustituyentes, se realiza la menor numeración posible.
	\item Si los sustituyentes tienen la misma posición al enumerar la cadena en forma horaria o antihoraria, la menor numeración le corresponde al grupo alquilo que se nombra primero según el orden alfabético.
	\item Si el grupo alquilo tiene mayor número de átomos de carbono que el anillo, entonces la cadena cíclica será considerada como sustituyente.
	\item Si hay halógenos en el cicloalcano estos no tiene ninguna prioridad sobre los grupos alquilo, solo se considera el orden alfabético.
\end{enumerate}
\noindent Ejemplo: \\
\begin{gather*}
	\chemfig{*4(--(-\ch{CH2CH3})--)} \\
	\iupac{Etilciclobutano} \\
	\chemfig{*4(--(-\ch{CH2CH2CH2})--)} \\
	\iupac{Propilciclobutano} \\
	\chemfig{*5(---(-*3(---))--)} \\
	\iupac{Ciclopropilciclopentano} \\
	\chemfig{*4(--(-[:75]\ch{CH3})(-[:15]\ch{CH3})--)} \\
	\iupac{1,1-Dimetilciclobutano} \\
	\chemfig{*5((-\ch{CH3})--(-\ch{CH2CH3})---)} \\
	\iupac{1-Etil-3-metilciclopentano} \\
	\chemfig{*6(---(-\ch{CH2CH2CH2CH3})-(-(-[:30])(-[:150]))--)} \\
	\iupac{1-Butil-2-isopropilciclohexano} \\
\end{gather*}
\begin{gather*}
	\chemfig{*5(--(-\ch{CH3})-(-\ch{CH3})-(-\ch{CH3})-)} \\
	\iupac{1,2,3-Trimetilciclopentano} \\
	\chemfig{*6(--(-\ch{CH3})--(-\ch{CH2CH3})--(-\ch{CH2}-\ch{CH2}-\ch{CH3})-)} \\
	\iupac{1-Etil-3-metil-5-propilciclohexano} \\
	\chemfig{CH3-CH(-[2]*3(---))-CH2-CH2-CH3} \\
	\iupac{2-Ciclopropilpentano} \\
	\chemfig{-[:-30](-[-2])-[:30]-[:-30](-[-2]([:-90,0.7]*5(-----)))-[:30]-[:-30]-[:30]-[:-30]} \\
	\iupac{4-Ciclopentil-2,2,6-trimetilheptano}
\end{gather*}
\subsection*{Oxigenados}
\subsubsection*{Alcoholes}
\begin{Theorem*} {Al}
	\begin{gather*}
		\text{Formulación molecular} \\
		\chemfig{C_nH_{2n}} \\
		\text{Formulación desarrollada} \\
		\chemfig{*3(---)} \\
		\text{Formulación desarrollada} \\
		\text{prefijo: "ciclo" \quad sufijo: "ano"}
	\end{gather*}
\end{Theorem*}
\noindent Ejemplo: \\
\begin{gather*}
	\chemfig{C(-[:150]H)(-[:210]H)*3(-C(-[-2]H)(-[:-30]H)-C(-[2]H)(-[:30]H)-)} \\
	\iupac{Ciclopropano} \\
	\chemfig{C*6(-C-C-C-C-C-)}\\
	\iupac{Ciclohexano}
\end{gather*}
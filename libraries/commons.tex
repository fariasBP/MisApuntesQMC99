\usepackage[utf8]{inputenc}
\usepackage{amsmath}
\usepackage{amssymb}
\usepackage{amsfonts}
\usepackage[left=0.8cm,right=0.8cm,top=0.8cm,bottom=0.8cm]{geometry}
\usepackage{lipsum}
\usepackage{xcolor}
\usepackage{sectsty}
\usepackage{multicol} % para columnas
\usepackage{tcolorbox} % para cajas
%\usepackage[all]{xy} % para matrices con flecha (usada para el metodo aspa entre otros)

%\usepackage{physics} % para derivadas variables fisicas etc.

% definiendo importador de imagenes
\usepackage{float}
\usepackage{graphicx} % para importar imagenes
\usepackage{cancel} %para cancelar, tachar expresiones (util para la notacion de simplificar)
\usepackage{chemmacros} % simbologia quimica https://ctan.math.washington.edu/tex-archive/macros/latex/contrib/chemmacros/chemmacros-manual.pdf
	\chemsetup[redox]{explicit-sign = true}
	\chemsetup[redox]{roman = false}
	\chemsetup[redox]{pos=top}
\usepackage{chemfig} %para nomeclatura organica
\usepackage{tabularx} % para que tablas u otros ocupen todo el ancho disponible
\usepackage{multirow} % para crear celdas que tomen muchas filas en tablas 

% para tabla periodica
\usepackage{tikz} % para crear objetos
\usepackage{hyperref}
\usetikzlibrary{shapes}